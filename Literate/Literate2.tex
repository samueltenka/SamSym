\documentclass{article}\usepackage{amsmath}\begin{document}
\title{Literate Programming System}
\maketitle

This is a program to help write programs, in particular to organize them in a
readable way. The idea is that we write programs in a freer format than normal,
then feed it into this program which outputs an ``illiterate" or ``expanded"
version. As a bonus, it also processes comments such as the one currently being
read, to produce beautiful documentation. The idea is due to Donald Knuth, and I
love it.
In particular, there exists a program $E$ that eats input files $I$ and outputs
an expanded output $O$ in accordance with a makefile $M$: $E_M(I)=O$. We also
have a program $D$ that outputs documentation: $D_M(I)=L$. Then $E$ is
``described" by this program $I^\star$, in the sense that, for some $M^\star$,
$E_{M^\star}(I^\star)=E$; or, if you are reading a nice .tex or .pdf $L^\star$
rather than weird-looking python-markdown hybrid, in the sense that there exist
$I^\star, M^\star$ as above, such that $D_{M^star}(I^star)=L^\star$ too.
We create a ``web" --- or at least a tree --- of program snippets.
\section{Bird's Eye View}
The global structure of the program is to slurp a makefile, then learn from the
input files specified therein, then expand those into an output file. Note that
special characters ``<<<", ``>>>", etc. are not misinterpreted when mentioned as
string literals: indeed, our literate programming tags must be the sole
occupants of their line, aside from spaces and tabs. Whitespace matters.
\begin{verbatim}
definitions = {}; tex_body = ''
DEF_BEG, DEF_END, EXP_BEG, EXP_END = '<'*3, '>'*3, '<'+'@', '@'+'>'
START_MACRO = 'START'; MAKE_FILE_NAME = 'make.txt'
<@ Orient according to makefile @>
<@ Learn from input files @>
<@ Expand into outputs @>
\end{verbatim}
\subsection{Orient}

Makefiles have format:
LINE 0> output filename
LINE 1> input filename \#1
. . .
LINE N> input filename \#N
LINE N+1> documentation filename
\begin{verbatim}
filenames = []
with open(MAKE_FILE_NAME) as file:
   filenames = file.read().split('\n')
out, ins, doc = filenames[0], filenames[1:-1], filenames[-1]
\end{verbatim}
\subsection{Learn}

\begin{verbatim}
for filename in ins:
   with open(filename) as file:
      text = file.read()
      <@ Learn from text @>
\end{verbatim}
\subsection{Expand}

\begin{verbatim}
<@ Expand to text @>
<@ Craft documentation @>
\end{verbatim}
\section{Learn}

\begin{verbatim}
while '\n\n\n' in text:
   text = text.replace('\n\n\n', '\n\n')
sections = text.split('\n\n')
for section in sections:
   lines = section.split('\n')
   strp = lines[0].strip()
   <@ macro definition case @>
   <@ commentary case @>
   <@ commentary continuation case @>
\end{verbatim}
\begin{verbatim}
if len(strp)>=6 and strp[:3]==DEF_BEG and strp[-3:]==DEF_END:
   name = strp[3:-3].strip()
   <@ ensure macro not already defined @>
   definitions[name]='\n'.join(lines[1:])
   <@ document macro definition @>
\end{verbatim}
Note that, though subheadings only have above underlines, they must still have
at least one free line underneath.
\begin{verbatim}
elif strp[:5] in ['=====','-----','_____']:
   if strp[:5]=='=====':
       <@ make title @>
   elif strp[:5]=='-----':
       <@ make section @>
   elif strp[:5]=='_____':
       <@ make subsection @>
   lines = lines[3:]
   <@ make body @>
\end{verbatim}
\begin{verbatim}
else:
   <@ make body @>
\end{verbatim}
\subsection{Typesetting}

\begin{verbatim}
tex_body += '\\'+'title{'+lines[1]+'}\n' +\
            '\\'+'maketitle\n'
\end{verbatim}
\begin{verbatim}
tex_body += '\\'+'section{'+lines[1]+'}\n'
\end{verbatim}
\begin{verbatim}
tex_body += '\\'+'subsection{'+lines[1]+'}\n'
\end{verbatim}
\begin{verbatim}
tex_body += '\n'.join(lines)+'\n\n'
\end{verbatim}
\begin{verbatim}
tex_body += '\\'+'begin{verbatim}\n'+\
            definitions[name]+'\n'+\
            '\\'+'end{verbatim}\n'
\end{verbatim}
\subsection{Minor detail}

One very powerful aspect of literate programming, we've found, is to reduce the
clutter of gaurds such as:
\begin{verbatim}
if name in definitions:
   print('may not redefine macro', name)
   exit(-1)
\end{verbatim}
\section{Expand}

\subsection{Create Illiterate Output Program}

\begin{verbatim}
text = definitions[START_MACRO]
<@ while some lines are expandable, for each line: @>
      <@ expand if the line's expandable @>
<@ write expanded to file @>
\end{verbatim}
\begin{verbatim}
while EXP_BEG in text:
   lines = text.split('\n')
   text = ''
   for line in lines:
\end{verbatim}
\begin{verbatim}
if EXP_BEG in line:
   <@ get macro name, spacing @>
   <@ ensure macro is defined @>
   <@ replace macro name by definition @>
else:
    text += line+'\n'
\end{verbatim}
\begin{verbatim}
spacing = line[:len(line)-len(line.lstrip())]
name = line.replace(EXP_BEG, '').replace(EXP_END, '').strip()
\end{verbatim}
\begin{verbatim}
if name not in definitions:
   print('macro not found:', name)
   exit(-1)
\end{verbatim}
\begin{verbatim}
for defline in definitions[name].split('\n'):
   text += spacing + defline + '\n'
\end{verbatim}
\begin{verbatim}
with open(out, 'w') as f:
   f.write(text)
\end{verbatim}
\subsection{Create Documentation}

\begin{verbatim}
<@ assemble latex fragments @>
<@ write documentation to file @>
\end{verbatim}
\begin{verbatim}
tex_begin = '\\'+'documentclass{article}'+\
            '\\'+'usepackage{amsmath}'+\
            '\\'+'begin{document}\n'
tex_end =   '\\'+'end{document}'
tex = tex_begin+tex_body+tex_end
\end{verbatim}
\begin{verbatim}
with open(doc, 'w') as f:
   f.write(tex)
\end{verbatim}
\end{document}
