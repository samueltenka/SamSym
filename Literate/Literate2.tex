\documentclass{article}
\usepackage[usenames,dvipsnames]{xcolor}
\usepackage{amsmath, etoolbox}
\makeatletter
\preto{\@verbatim}{\topsep=0pt \partopsep=0pt }
\makeatother
\begin{document}
\title{Literate Programming System}
\maketitle

This is a program to help write programs, in particular to organize them in a
readable way. The idea is that we write programs in a freer format than normal,
then feed it into this program which outputs an ``illiterate" or ``expanded"
version. As a bonus, it also processes comments such as the one currently being
read, to produce beautiful documentation. The idea is due to Donald Knuth, and I
love it.

In particular, there exists a program $E$ that eats input files $I$ and outputs
an expanded output $O$ in accordance with a makefile $M$: $E_M(I)=O$. We also
have a program $D$ that outputs documentation: $D_M(I)=L$. Then $E$ is
``described" by this program $I^\star$, in the sense that, for some $M^\star$,
$E_{M^\star}(I^\star)=E$; or, if you are reading a nice .tex or .pdf $L^\star$
rather than weird-looking python-markdown hybrid, in the sense that there exist
$I^\star, M^\star$ as above, such that $D_{M^\star}(I^\star)=L^\star$ too.

We create a ``web" --- or at least a tree --- of program snippets.

\section{Bird's Eye View}The global structure of the program is to slurp a makefile, then learn from the
input files specified therein, then expand those into an output file. Note that
special characters ``<<<", ``>>>", etc. are not misinterpreted when mentioned as
string literals: indeed, our literate programming tags must be the sole
occupants of their line, aside from spaces and tabs. Whitespace matters.

$$\boxed{\text{START}}$${\color{YellowOrange}\begin{verbatim}definitions = {}; tex_body = ''
DEF_BEG, DEF_END, EXP_BEG, EXP_END = '<'*3, '>'*3, '<'+'@', '@'+'>'
START_MACRO='START'; MAKE_FILENAME='make.txt'; SNIPPETS_FILENAME='snippets.tex'
\end{verbatim}}{\color{MidnightBlue}\begin{verbatim}<@ Orient according to makefile@>
<@ Load tex snippets @>
<@ Learn from input files @>
<@ Expand into outputs @>
\end{verbatim}}\subsection{Orient}

Makefiles have format:
LINE 0> output filename
LINE 1> input filename \#1
. . .
LINE N> input filename \#N
LINE N+1> documentation filename

$$\boxed{\text{Orient according to makefile}}$${\color{YellowOrange}\begin{verbatim}filenames = []
with open(MAKE_FILENAME) as file:
   filenames = file.read().split('\n')
out, ins, doc = filenames[0], filenames[1:-1], filenames[-1]
\end{verbatim}}$$\boxed{\text{Load tex snippets}}$${\color{YellowOrange}\begin{verbatim}with open(SNIPPETS_FILENAME, 'r') as f:
   DOC_SNIPPET, \
   TITLE_SNIPPET, SECTION_SNIPPET, SUBSECTION_SNIPPET, \
   REGULAR_SNIPPET, EXPANDABLE_SNIPPET, \
   MACRO_BOX = f.read().split('\n\n')
\end{verbatim}}\subsection{Learn}

$$\boxed{\text{Learn from input files}}$${\color{YellowOrange}\begin{verbatim}for filename in ins:
   with open(filename) as file:
      text = file.read()
\end{verbatim}}{\color{MidnightBlue}\begin{verbatim}      <@ Learn from text @>
\end{verbatim}}\subsection{Expand}

$$\boxed{\text{Expand into outputs}}$${\color{MidnightBlue}\begin{verbatim}<@ Expand to text @>
<@ Craft documentation @>
\end{verbatim}}\section{Learn}

$$\boxed{\text{Learn from text}}$${\color{YellowOrange}\begin{verbatim}text = '\n'.join((line if line.strip() else '') for line in text.split('\n'))
while '\n\n\n' in text:
   text = text.replace('\n\n\n', '\n\n')
sections = text.split('\n\n')
for section in sections:
   lines = section.split('\n')
   strp = lines[0].strip()
\end{verbatim}}{\color{MidnightBlue}\begin{verbatim}   <@ macro definition case @>
   <@ commentary case @>
   <@ commentary continuation case @>
\end{verbatim}}$$\boxed{\text{macro definition case}}$${\color{YellowOrange}\begin{verbatim}if len(strp)>=6 and strp[:3]==DEF_BEG and strp[-3:]==DEF_END:
   name = strp[3:-3].strip()
\end{verbatim}}{\color{MidnightBlue}\begin{verbatim}   <@ ensure macro not already defined @>
\end{verbatim}}{\color{YellowOrange}\begin{verbatim}   definitions[name]='\n'.join(lines[1:])
\end{verbatim}}{\color{MidnightBlue}\begin{verbatim}   <@ document macro definition @>
\end{verbatim}}Note that, though subheadings only have above underlines, they must still have
at least one free line underneath.

$$\boxed{\text{commentary case}}$${\color{YellowOrange}\begin{verbatim}elif strp[:5] in ['=====','-----','_____']:
   if strp[:5]=='=====':
\end{verbatim}}{\color{MidnightBlue}\begin{verbatim}       <@ make title @>
\end{verbatim}}{\color{YellowOrange}\begin{verbatim}   elif strp[:5]=='-----':
\end{verbatim}}{\color{MidnightBlue}\begin{verbatim}       <@ make section @>
\end{verbatim}}{\color{YellowOrange}\begin{verbatim}   elif strp[:5]=='_____':
\end{verbatim}}{\color{MidnightBlue}\begin{verbatim}       <@ make subsection @>
\end{verbatim}}{\color{YellowOrange}\begin{verbatim}   lines = lines[3:]
\end{verbatim}}{\color{MidnightBlue}\begin{verbatim}   <@ make body @>
\end{verbatim}}$$\boxed{\text{commentary continuation case}}$${\color{YellowOrange}\begin{verbatim}else:
\end{verbatim}}{\color{MidnightBlue}\begin{verbatim}   <@ make body @>
\end{verbatim}}\subsection{Typesetting}

$$\boxed{\text{make title}}$${\color{YellowOrange}\begin{verbatim}tex_body += TITLE_SNIPPET.replace('TITLE', lines[1])
\end{verbatim}}$$\boxed{\text{make section}}$${\color{YellowOrange}\begin{verbatim}tex_body += SECTION_SNIPPET.replace('S HEADING', lines[1])
\end{verbatim}}$$\boxed{\text{make subsection}}$${\color{YellowOrange}\begin{verbatim}tex_body += SUBSECTION_SNIPPET.replace('SS HEADING', lines[1])
\end{verbatim}}$$\boxed{\text{make body}}$${\color{YellowOrange}\begin{verbatim}tex_body += '\n'.join(lines)+'\n\n'
\end{verbatim}}$$\boxed{\text{document macro definition}}$${\color{YellowOrange}\begin{verbatim}tex_body += MACRO_BOX.replace('MACRO NAME', name)
is_expandable = lambda line: EXP_BEG in line
lines = definitions[name].split('\n')
while lines:
\end{verbatim}}{\color{MidnightBlue}\begin{verbatim}   <@ regular block @>
   <@ expandable block @>
\end{verbatim}}$$\boxed{\text{regular block}}$${\color{YellowOrange}\begin{verbatim}if lines and not is_expandable(lines[0]):
   code = ''
   while lines and not is_expandable(lines[0]):
       code += lines[0]+'\n'
       lines = lines[1:]
   tex_body += REGULAR_SNIPPET.replace('REGULAR CODE', code)
\end{verbatim}}$$\boxed{\text{expandable block}}$${\color{YellowOrange}\begin{verbatim}if lines and is_expandable(lines[0]):
   code = ''
   while lines and is_expandable(lines[0]):
       code += lines[0]+'\n'
       lines = lines[1:]
   tex_body += EXPANDABLE_SNIPPET.replace('EXPANDABLE', code)
\end{verbatim}}\subsection{Minor detail}

One very powerful aspect of literate programming, we've found, is to reduce the
clutter of gaurds such as:

$$\boxed{\text{ensure macro not already defined}}$${\color{YellowOrange}\begin{verbatim}if name in definitions:
   print('may not redefine macro', name)
   exit(-1)
\end{verbatim}}\section{Expand}

\subsection{Create Illiterate Output Program}

$$\boxed{\text{Expand to text}}$${\color{YellowOrange}\begin{verbatim}text = definitions[START_MACRO]
\end{verbatim}}{\color{MidnightBlue}\begin{verbatim}<@ while some lines are expandable, for each line: @>
      <@ expand if the line's expandable @>
<@ write expanded to file @>
\end{verbatim}}$$\boxed{\text{while some lines are expandable, for each line:}}$${\color{YellowOrange}\begin{verbatim}while EXP_BEG in text:
   lines = text.split('\n')
   text = ''
   for line in lines:
\end{verbatim}}$$\boxed{\text{expand if the line's expandable}}$${\color{YellowOrange}\begin{verbatim}if EXP_BEG in line:
\end{verbatim}}{\color{MidnightBlue}\begin{verbatim}   <@ get macro name, spacing @>
   <@ ensure macro is defined @>
   <@ replace macro name by definition @>
\end{verbatim}}{\color{YellowOrange}\begin{verbatim}else:
    text += line+'\n'
\end{verbatim}}$$\boxed{\text{get macro name, spacing}}$${\color{YellowOrange}\begin{verbatim}spacing = line[:len(line)-len(line.lstrip())]
name = line.replace(EXP_BEG, '').replace(EXP_END, '').strip()
\end{verbatim}}$$\boxed{\text{ensure macro is defined}}$${\color{YellowOrange}\begin{verbatim}if name not in definitions:
   print('macro not found:', name)
   exit(-1)
\end{verbatim}}$$\boxed{\text{replace macro name by definition}}$${\color{YellowOrange}\begin{verbatim}for defline in definitions[name].split('\n'):
   text += spacing + defline + '\n'
\end{verbatim}}$$\boxed{\text{write expanded to file}}$${\color{YellowOrange}\begin{verbatim}with open(out, 'w') as f:
   f.write(text)
\end{verbatim}}\subsection{Create Documentation}

$$\boxed{\text{Craft documentation}}$${\color{YellowOrange}\begin{verbatim}tex = DOC_SNIPPET.replace('BODY', tex_body)
with open(doc, 'w') as f:
   f.write(tex)
\end{verbatim}}
\end{document}
